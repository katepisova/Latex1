\documentclass[11pt,a4paper]{article}
\usepackage[left=30mm, right=15mm, top=20mm, bottom=20mm]{geometry}
\pagestyle{empty}
\usepackage{titlesec}
\renewcommand{\thesubsection}{\arabic{section}.\arabic{subsection}.}
\renewcommand{\thesection}{\arabic{section}.}
\usepackage[utf8]{inputenc}
\usepackage{indentfirst}
\setlength{\parindent}{1.2cm}
\setlength{\parskip}{0.2cm}
\usepackage[T2A]{fontenc}
\usepackage{epigraph}
\usepackage{ragged2e}
\setlength{\epigraphwidth}{0.57\textwidth}
\setlength{\epigraphrule}{0pt}
\usepackage[english,russian]{babel}
\usepackage{enumitem}
\setlist{nolistsep, itemsep=6pt, leftmargin=0mm, itemindent=14mm}
\addto\captionsrussian{\def\refname{Список литературы}}
\begin{document}
\begin{center}
\huge{\textbf{Сложность}}
\end{center}
\epigraph
{\justifying\small
  Врач, строитель и программистка спорили о том, чья профессия древнее. Врач заметил: "В Библии сказано, что Бог сотворил Еву из ребра Адама. Такая операция может быть проведена только хирургом, поэтому я по праву могу утверждать, что моя профессия самая древняя в мире". Тут вмешался строитель и сказал: "Но еще раньше в Книге Бытия сказано, что Бог сотворил из хаоса небо и землю. Это было первое и, несомненно, наиболее выдающееся строительство. Поэтому, дорогой доктор, вы не правы. Моя профессия самая древняя в мире". Программистка при этих словах откинулась в кресле и с улыбкой произнесла: "А кто же по-вашему сотворил хаос?"}
\justifying
\section[14pt]{Сложность, присущая программному обеспечению}
\subsection[13pt]{Простые и сложные программные системы}
\subsection[13pt]{Почему программному обеспечению присуща сложность?}


\subsection[13pt]{Последствия неограниченной сложности}
 

\section[14pt]{Структура сложных систем}
\subsection{Примеры сложных систем }


\subsection{Пять признаков сложной системы }


\addcontentsline{toc}{section}{Список литературы}
\begin{thebibliography}{}
    \bibitem{litlink1} Brooks, F. April 1987. No Silver Bullet: Essence and Accidents of Software Engineering. IEEE Computer vol.20(4), p.12. 
    \bibitem{litlink2} Peters, L. 1981. Software Design. New York, NY: Yourdon Press, p.22. 
    \bibitem{litlink3} Brooks. No Silver Bullet, p.11
    \bibitem{litlink4} Parnas, D. July 1985. Software Aspects ofStrategic Defense Systems. Victoria, Canada: University of Victoria. Report DCS-47-IR. 
    \bibitem{litlink5} Peter, L. 1986. The Peter Pyramid. New York, NY: William Morrow, p.153. 
    \bibitem{litlink6} Waldrop, M. 1992. Complexity: The Emerging Science at the Edge of Order and Chaos. New-York, NY: Simon and Schuster. 
    \bibitem{litlink7} Courtois, P. June 1985. On Time and Space Decomposition of Complex Structures. Communications of the ACM vol.28(6), p.596. 
    \bibitem{litlink8} Simon, H. 1982. The Sciences of the Artificial. Cambridge, MA: The MIT Press, p.218. 
    \bibitem{litlink9} Rechtin, E. October 1992. The Art of Systems Architecting. IEEE Spectrum, vol.29( 10), p.66. 
    \bibitem{litlink10} Simon. Sciences, p.217. 
    \bibitem{litlink11} Ibid, p.221
\end{thebibliography}
\end{document}
